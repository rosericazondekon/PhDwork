%*******10********20********30********40********50********60********70********80

% For all chapters, use the newdefined chap{} instead of chapter{}
% This will make the text at the top-left of the page be the same as the chapter

%\chap{Literature Review}
\chap{A review of the related literature on disease research and applications of network analysis}

\section{Brief Overview of Malaria, Tuberculosis and HIV/AIDS}
AIDS is a health condition caused by the Human Immunodeficiency Virus (HIV) \cite{mbulaiteye_hiv_2011,dalessandro_comparison_1995}. HIV infects and attacks the cells that are responsible for the immune system in the body (CD4 cells) that provide protection against infections and illness. The virus infects the human host by making him vulnerable and unable to fight future infections \cite{joint_united_nations_programme_on_hiv/aids_aids_2007}. The virus eventually weakens and kills the CD4 cells resulting in a weak immune system and vulnerability to diseases. HIV is transmitted through body fluids exchange, and the infection exists in four stages. The first stage is the primary infection stage and lasts within 2 to 4 weeks. It is characterized by flu-like symptoms, and the infected person is highly contagious. The second stage is the asymptomatic stage that may last for about ten years, and the infected person does not display significant symptoms of the infections. The third stage is the symptomatic stage. At this stage, the virus weakens the immune system, and the infected person suffers from both mild and chronic symptoms as the infected person suffers opportunistic diseases. Illnesses like malaria and TB in HIV infected subjects, are experienced in a severe manner. The fourth stage is AIDS; it causes death within two years if left untreated \cite{joint_united_nations_programme_on_hiv/aids_aids_2007,whiteside_hiv/aids:_2008}.\\
According to the World Health Organization (WHO) the signs for HIV/AIDS change through the stages of infections as the disease progresses. To determine whether a person is infected, an HIV test needs to be conducted. ELISA method based HIV testing is one of the most common antibody-based testing method characterized by 99\% accuracy rate \cite{mbulaiteye_hiv_2011}. It is recommended that a HIV negative test result should be confirmed after three months because the immune system can sometimes take up to 12 weeks to develop the tested antibodies \cite{centers_for_disease_control_and_prevention_revised_2007}. It is however possible to get false negative results during the 12 weeks window period. The antiretroviral (ARV) drug therapy is initiated when the infected person reaches the third or fourth stage of infection to suppress the virus and boost the immune system. Such measures are taken because there is currently, no cure for HIV and the early initiation of the therapy may result in drug resistance \cite{brenner_we_2016,calmy_hiv_2004,clavel_hiv_2004}.\\
Unlike HIV/AIDS, TB is a highly infectious disease that is caused by a bacteria called \textit{Mycobacterium tuberculosis}. The disease exists in active and inactive forms. The active form, also known as the open disease causes the infected person to suffer and to be highly infectious. The inactive/latent TB infection is not infectious, and the infected individual does not suffer from the signs and symptoms associated with the active disease. Healthy individuals with latent infection have approximately 10\% probability of getting active TB disease over their life. Chances of infection are high in the first two years after the exposure to the bacteria, and in the case where the host develops any form of lung or immune system damage \cite{kaufmann_handbook_2008,zumla_handbook_2009}. On the other hand, in HIV infected individuals co-infected with TB, there exists a 10\% annual chance of developing active TB \cite{raviglione_tuberculosis_1997,sharma_hiv-tb_2005,toossi_impact_2001}. Active TB in adults may result from re-infection with a new strain of TB or perhaps a reaction to the latent infection. Consequently, researchers surmise that silica inhalation, HIV infection, and silicosis are responsible for the high risk of TB infection in the working adults’ population \cite{sharma_hiv-tb_2005,danibrosio_epidemiology_2014}.
TB symptoms are characterized by a chronic cough, night-time fevers, profuse sweating, and significant weight loss within a short time. However, studies show that people with TB can be infectious prior to showing the symptoms or complaining of any form of pulmonary discomfort. In the worst case scenario, TB goes beyond the pulmonary and infects other parts of the body, especially for people infected with HIV. HIV complicates the manifestation of TB in terms of its symptoms and signs in 70\% of the HIV/AIDS infected population suffering from TB \cite{sharma_hiv-tb_2005}. Studies indicate that people with undetected open TB disease are the leading cause of TB infections. Even though TB is a treatable disease, the treatment procedure is extremely aggressive. The treatment procedure for first-time patients entails administration of a six-months dose under close medical supervision termed as directly observed therapy. The other challenge in the treatment is that there are approximately 25\% TB-drugs resistance cases worldwide every year \cite{centers_for_disease_control_and_prevention_cdc_emergence_2006,world_health_organization_multidrug_2010}. Approximately 80\% of people with TB can be cured of their active TB infection, however, HIV and Silicosis increases the risk of reinfection by 20\%. The infection among individuals with silicosis, may cumulatively contribute to lung damage and work inability. Additionally, the HIV/AIDS increases the risk of opportunistic infections, which may result in a poor outcome for the TB treatment \cite{cowie_epidemiology_1994,mulenga_silicosis_2013,rees_silica_2007}.\\
Completing the trifecta is malaria, a parasitic infectious disease caused by the Plasmodium parasites. Even though, malaria is predominantly found in the tropical regions, 48\% of the instances of infections have been experienced in the Northern and Southern parts of America, Asia, and Africa, putting approximately 50\% of the world’s population at risk. The malaria pathogens are \textit{Plasmodium ovale}, \textit{Plasmodium malariae}, \textit{Plasmodium vivax}, and \textit{Plasmodium falciparum} which is the deadliest. The distribution of the disease matches that of its vectors, the female mosquitoes of the genus Anopheles \cite{sinka_global_2012,snow_global_2005}. In the sub-Saharan African countries, the vector of the disease is \textit{Anopheles gambiae s.l}. Malaria has a range of symptoms and signs that manifest differently from one person to another. The most common symptoms are fevers, gastrointestinal symptoms, and fatigue, headaches, and muscle aches. The malaria pathogen infects two hosts, the Anopheles mosquito, and the infected human. When the infected mosquito feeds from an individual, it injects sporozoites into the circulatory system of the bitten person. The sporozoites reside in the liver cells until they become mature schizonts. The schizonts rupture upon maturity and release merozoites, which infect the red blood cells \cite{james_new_1937}. The two most used malaria test are rapid tests using an instant result kit akin to the home pregnancy test device, and the blood smear test that is examined under the microscope for the presence of red blood cells that are infected by the parasite. Treatment entails administration of drugs that range in types. While some malaria drug prescriptions may have a three days dosage, others may have up to one week dosage \cite{alonso_malaria_1993,battle_treatment-seeking_2016}.\\
HIV/AIDS, TB, and Malaria form together a trifecta of diseases caused respectively by a virus, a bacteria, and a parasite.

\section{Network Analysis of Scientific Research collaboration}
Collaboration in science is essential to research and development, knowledge discovery, technology and innovation. 
%It occupies a predominant place in scientometrics. 
The effectiveness of collaboration in science can be measured using scientometrics. According to Leydesdorff and Milojevic \cite{leydesdorff_scientometrics_2012}, scientometrics uses quantitative and computational methods to analyzing and measuring science, communication in science and science policy. %According to the same authors, 
The field of scientometrics emerged from Eugene Garfield’s idea to improve Information Retrieval \cite{eugene_citation_1979}, followed by the creation of the Science Citation Index (SCI) in the 1960s, and the availability of scientific databases references publications. The discipline of Scientometrics is aimed at providing guidance to several research issues involving the measurement of science impact, the measurement of impact journals and institutional units, theories of citation, and the mapping of science. Here, we focus on the mapping of science since it is essential to understanding the dynamic of science, informing policy decisions, and identifying important fields, research groups, as well as specialties based on evidence from the literature \cite{leydesdorff_scientometrics_2012}. Such goals can be achieved by mapping publications, authors and analyzing patterns of collaborations between them.\\
Since the publication of the first co-authored paper in 1665, scientific co-authorship has spread significantly throughout the scientific realm and the number of co-authored scientific publications have tremendously increased \cite{luukkonen_understanding_1992}. According to Wagner \cite{wagner_six_2005}, the increase in international scientific co-authorship has been of a fast growth. International co-authorship originates from international collaborations between scientists. In general, international collaborations have more visibility than national collaborations and often result in publications in high impact journals \cite{glanzel_analysing_2004}.\\
The paradigm of co-authorship network is rooted in network theory. In a co-authorship network, the reasearchers are represented by the set of vertices and the relationship between them are represented by the set of edges. An edge between two researchers in such a network means that they both coauthor a publication. Unlike citation networks, the scientific community has dedicated less attention to co-authorship networks because of the long tradition of citation network analysis in bibliometric \cite{newman_structure_2001,newman_coauthorship_2004}. Nevertheless, the analyses of how complex co-authorship networks form and evolve in time is crucial for identifying leading researchers in a particular scientific domain, describing their extant to collaborate with their peers, and evaluating the impact of their research \cite{gonzalez-alcaide_scientific_2012}. An example of such an investigation is illustrated in Newman scientific collaboration paper series on Biomedical research, physics and computer science co-authorship networks \cite{newman_structure_2001,newman_coauthorship_2004,newman_scientific_2001,newman_scientific_2001-1}.\\
Taking publications as units, the analyses of scientific collaboration facilitate the study of trans and inter-disciplinary research by focusing on the dynamics of the collaboration networks \cite{borner_visualizing_2003}.  In addition, these networks can provide important information regarding cooperation patterns among authors and their status and location in the structures of the scientific community \cite{scharnhorst_models_2012}. Furthermore, Mali et al. \cite{mali_dynamic_2012} assert that co-authorship social network studies are highly relevant for funding organizations for promising and emerging topics support in science.\\
Although many authors have proposed different features for classifying co-authorship networks \cite{andrade_dimensions_2009,rogers_obstacles_2001,sonnenwald_scientific_2007}, the categorization features of Andrade et al. \cite{andrade_dimensions_2009} identifies three levels of classification of scientific collaboration: the cross-disciplinary level with the intradisciplinarity and interdisciplinarity subdimensions, the cross-sectoral level with the intramural and extramural research collaboration subdimensions and the cross-national level including the national and international scientific collaboration subdimensions. For a full description of each level of scientific collaboration, we refer the reader to Mali et al. \cite{mali_dynamic_2012}. \\%Our work focuses more on the intradisciplinarity and interdisciplinarity subdimensions of the cross-disciplinary level.\\
The methods of co-authorship network studies have emerged from social network analysis and graph theory. Such studies heavily relied upon access to scientific collaboration data sources such as SCOPUS, the Web Of Science, PubMed, Medline or even Google Scholar. In general, Mali et al. \cite{mali_dynamic_2012} identify three methodological approaches to studying scientific co-authorship networks:%\\
\begin{displayquote}
(i) basic analysis of network properties using temporal data (usually in the form of a time-series of snapshots), (ii) deterministic approaches to the analysis of scientific co-authorship networks, and (iii) statistical modeling of network dynamics
\end{displayquote}
%Despite the identification of the three methodological approaches, an important body of the literature involving co-authorship network studies rely on the basic analysis of network properties such as 
In addition to the three approaches outlined by Mali et al. \cite{mali_dynamic_2012} and mentioned above, co-authorship networks can be analyzed on the basis of formal network properties, including network degree, density, path, path length, shortest path and the global clustering coefficient. Many scientific collaboration network studies have adopted this graph-based approach to scientific co-authorship investigation. In the next paragraphs, we present and discuss the purpose, methods and the results of some of those studies.\\
Newman \cite{newman_structure_2001} investigated scientific network collaboration in biomedical research, physics and computer science. In this study, Newman collected data from four databases and presented distribution of collaboration networks, demonstrated the presence of clustering and highlights differences between the scientific fields under investigation. According to his findings, Newman \cite{newman_structure_2001} concluded on the "smallworldness" of such networks in which scientists are only separated by shorter paths. In a second paper published the same year, Newman \cite{newman_scientific_2001} provided a deeper analysis of the networks using the same data. He presented a variety of statistical properties of the networks, identified giant collaborative components and study centrality and connectedness measures. In Newman \cite{newman_scientific_2001-1}, the author evaluated various nonlocal network properties including shortest paths and distance between researchers. He proposed a modified version of the standard breadth-first search algorithm for evaluating the geodesic distance between the scientists in the network. He later weighted the networks by the number of paper published by pairs of researchers as well as their number of coauthors, and calculated all the distances using Dijkstra's algorithm. His analyses provided insights in the strength of collaboration in each network. In a last paper in the same series, the author summarized the results of the three previous studies and showed how patterns of collaboration varied between scientists within a scientific field over time \cite{newman_coauthorship_2004}. \\
In another study, Hou et al. \cite{hou_structure_2008} applied a variety of graph-based algorithms to quantify the importance and impact of science, analyzing data retrieved from the Science Citation Index (SCI) over a period expanding from 1978 to 2014. In addition to methods of Social Network Analysis (SNA), the authors used co-occurrence analysis, cluster analysis and frequency analysis of words to describe the microstructure of the scientometrics network, revealing the major collaborative clusters and identifying the center of the scientometrics collaborative network. All analyses were performed using a free online software called Bibexcel and visualizations were displayed using the Pajek program. Similarly, to Newman’s publications, this paper applied basic network analysis based on network properties such as degree, closeness and betweenness centrality measures. Unlike Newman’s studies, it also accounted for citation data. Yet another paper reported the collaborative patterns in co-authorship network in the scientific discipline of reproductive biology \cite{gonzalez-alcaide_coauthorship_2008}. This study conducted a bibliometric analysis on 4,702 papers published in the field from 2003 to 2005. Although their analysis was basic, the study did not make use of any network property measures but was rather, mainly descriptive. Nevertheless, the study identified important components by applying an unspecified clustering algorithm using the Bibliom\'etricos software, and the Pajek program for data visualization. A similar bibliometric analysis is also reported by Toivanen and Ponomariov \cite{toivanen_african_2011} who investigated the research collaboration patterns in the African regional systems. Their data were publication records from African institutions from 2005 to 2009, processed via a proprietary text mining software named VantagePoint. Analysis of the network was performed using the UCINET software. The authors adopted an empirical clustering method based on the geographic regions within the African research context. Their research uncovered the dynamic nature of African collaborative efforts despite the lack of research capabilities, the structural weaknesses, and the uneven integration of resources. \\%South Africa proved to be the emerging hub as it holds critical network function for collaborative research in the African context.\\
Some researchers have studied scientific network co-authorship across a scientific discipline in specific institutions or organizations. For example, Bellanca \cite{bellanca_measuring_2009} used basic network analysis to measure interdisciplinary research by describing three co-authorship networks of researchers in Biology and chemistry departments at the University of York. After extracting publication records from the Web Of Science, the author used the Bibexcel tool and the UCINET software to analyze the co-authorship networks. The analysis was descriptive involving the assessment of basic network properties such as node degree, betweenness, and clustering coeeficient. They discovered fewer interdisciplinary research between biologists and chemists within the University but more interdisciplinary links between biology and mathematics, bioinformatics, biophysics and biochemistry. Their findings are potentially important for the development of strategies to promote interdisciplinary research within the University. Another study conducted in a Spanish institution analyzed collaboration between Spanish authors \cite{aleixandre-benavent_coauthorship_2008}. After retrieving 448 published papers between 1998 and 2007, the authors used basic network analysis, implemented in the Pajek program, to their network and identify group of authors as well as their relationship with others. In their future directions, the authors recommended that a dynamic time series analysis method as the next step to better understand their co-authorship network. %\\
In some other studies, the research focus was on a single country, across a specific scientific discipline. \\
Using Bibexcel, and the UCINET software package, Ghafouri et al. \cite{ghafouri_social_2014} proposed a sociogram analysis to social co-authorship network of Iranian researchers, in an attempt to help improve research prioritization, research centers establishment, teams and new curricula in the field of emergency medicine. Their results revealed a poorly connected, loose and sparse co-authorship network in the field of emergency medicine in Iran. While their study was keyword based and might have not included all papers, they recommended the rethink of research prioritization, the establishment of new research centers more emergency medicine specialists to Iranian policy makers. Yet another Iranian study by Salamati \& Soheili \cite{salamati_social_2016} investigated the field of violence, assessing scientific research outputs by Iranian researchers extracted from the Science Citation Index Expanded (SCIE), PubMed and Scopus databases, and covering the period 1972 to 2014. The authors used a combination of tools including Ravar Matrix, NetDraw to map coauthorship networks and VOSViewer, a software to draw co-word maps. Using basic network properties such as closeness, betweenness, eigenvector centrality measures, they identified structural holes, active authors, analyzed the structural indices of their network and evaluated the trend of published articles. One important limitation of their study was the attempt to manually standardize Iranian authors’ names and the keyword based search leading to the lack of comprehensiveness of the search results. \\
A similar study of Iranian researchers on Medical Parasitology  using NetDraw and the UCINET software package was also reported by Sadoughi et al. \cite{sadoughi_social_2016}. The study used basic network analysis to identify prolific researchers in the field of Medical parasitology by collecting 1048 published documents of all types in the field from 1972 to 2013 from the Web Of Science. The study identified aspects of scientific collaborations to help policy makers in the medical parasitology research area. A Brazilian study reported in the literature used the same methodological approach to generate new tools to help the Brazilian research fund to better select and prioritize research proposals \cite{morel_co-authorship_2009}. Publication records were collected from the Web Of Knowledge (also known as Web Of Science) scientific database on seven neglected tropical diseases. Co-authorship networks were generated for each disease and analyzed using Pajek and NetDraw, a tool of the UCINET software package. The text-mining was implemented using the VantagePoint software. The results generated new information leading to better design and strategic planning and implementation of a research funding program. This study further supports that traditional criteria to fund research such as research productivity or impact factor of scientific journals are not valuable indicators for grant selection in low productivity neglected tropical diseases research areas. This Brazilian study is one of the few that focused on co-authorship network in the fields of neglected tropical diseases and the vast field of tropical infectious disease. \\
In an attempt to promote cooperative and translational research initiatives, another study investigated the state of scientific collaboration on Chagas disease research \cite{gonzalez-alcaide_scientific_2012}. The study presented the analyzis of the scientific literature on Chagas disease published in the PubMed database between 1940 and 2009. On a total of 13,989 documents retrieved, the authors applied bibliometrics, social network analysis, and clustering methods implemented via the Pajek program to analyze the evaluation of collaboration patterns and to identify influential research groups. The results revealed a dramatic increase in research collaborations. As in Newman \cite{newman_structure_2001}, this study concluded that the co-authorship network of Chagas disease constitutes a "small world" network characterized by a high degree of clustering. Another important remark is the scarcity of African co-authorship network studies. Our review only identified the study by Toivanen and Ponomariov \cite{toivanen_african_2011} who focused on research collaboration patterns in the African regional systems with less insights into specific research areas.\\
The majority of the studies reviewed above implemented their analyses using the Pajek program \citep{BatageljPajek2014} or the UCINET software package which has the built-in NetDraw tool \citep{BorgattiUCINET2014}. The Pajek program is suitable for the analysis and visualization of large networks. It has Graphical user interface and has other features including multidiemnsional scaling and structural analysis. Unlike the Pajek program, the UCINET software package has built-in advanced features and can handle networks which size up to 10,000 nodes, and accepts a large number of network file format including the pajek format. In their entirety, the studies reviewed above applied descriptive, basic social analysis methods.\\
Recently, Zhang \cite{zhang_complex_2014} proposed a complex approach to social network analysis, emphasizing only on link prediction, one of the network topology inference questions. Her approach involved the development of a computationally efficient solution based on machine learning techniques such as naive bayes, support vector machine, K-nearest neighbor implemented in the data mining software Weka \citep{HallWEKAdatamining2009} and the Python package Scikit \citep{PedregosaScikitlearnMachinelearning2011}. The approach was tested on different datasets including a citation network, a co-authorship network and a protein-protein network. Quite often, these methods are not perfect since they failed to correctly tease out unreliable nodes from reliable ones, compromising the reliability of the network. However, new methodological approaches to scientific co-authorship network analysis are emerging to address those limitations. For example, Oliveira et al. \cite{oliveira_bayesian_2017} proposed a Bayesian approach to the analysis of such networks. Yet another limitation worth noting is that none of the studies reviewed above applied dynamic network analyses such as dynamic time series analysis or longitudinal network analysis \cite{mali_dynamic_2012}.%\\
%There has been a steady increase in published sources relating to Malaria TB, and HIV/AIDS. The increasing data sources have successfully contributed to accurate estimates and in-depth understanding of the trends of the diseases that have provided grounds to validate the global funding to fight TB, HIV/AIDS, and malaria. In the year 2000, a Global Fund was set up to fight the three diseases. The fund was administered by a non-governmental organization established by the World Health Organization (WHO). Ever since the year 2002 to 2016, these organizations have invested approximately \$19 billion in the control of the three infectious diseases. The investment saved 4.9 million lives. In the year 2008, the USA department of health under President Bush introduced the president’s malaria initiative \cite{stoops_presidents_2008} that provided \$1.2 million funding in a bid to reduce malaria-related deaths in sub-Saharan Africa. The effort included the provision of top-notch malaria treatment and prevention services in the highly affected African nations.

\section{Visualization tools for Co-authorship Networks}
Various authors have proposed diverse tools for specifically visualizing and exploring co-authorship network data. One of such tools has been reported by Liu and colleagues \cite{liu_toolkits_2004} who proposed an author navigator application for visual examination of co-authorhip networks. In their conception of the toolkits, the authors combined a web based application tool for the interactive navigation of the network and a Java based backend swing application for the management of CGI requests. To support Brazilian researchers, Barbosa and colleagues proposed \textbf{VRRC}, a web based tool for the visualization and recommendation of co-authorship network \cite{barbosa_vrrc:_2012}. According to its developers, \textbf{VRRC} provides an interactive visualization, an overview of the collaborations over time, and recommendations to initiate new collaborations and reinforce existing ones. \textbf{VICI}, another co-authorship visualization tool was proposed by  Odoni and colleagues \cite{odoni_visualisation_2017}. \textbf{VICI} combined a Python based backend system for the extraction and management of the network data and a web based frontend using Flask \cite{grinberg_flask_2014} to display the network. The visualization of the network was finally rendered using the Javascript D3.js \cite{bostock_d3._2012} library. \textbf{NeL$^2$}, a general purpose tool for the visualization of networks as a layered network diagram was proposed by Nakazono, Misue, and Tanaka \cite{nakazono_nel_2006}. They applied their tool to the visualization of co-authorship networks to visualize transitions in the network over a period of time, as well as various co-authorship data. \\
Another framework, the WebRelievo system was proposed for the visualization of the evolutionary processes of Web pages \cite{toyoda_system_2005}. Other techniques were also proposed for the visualization of co-citation networks \cite{chen_visualizing_1999}, and for the visualization of the relationship of scientific literature \cite{erten_simultaneous_2005}. In addition to their inability to display large networks, those proposed tools are limited by their lack of interactivity and their inability for the end user to easily control the display. We therefore could not just re-use any one of them in this dissertation.
%Here, we propose a co-authorship visualization and scientific collaboration tool for Malaria, TB and HIV/AIDS research in Benin. In addition to providing the same features as the aforementioned tools, we propose a different approach to co-authorship network visualization. Our approach integrates network structure and network data, hence requires no data management using traditional database framework. In addition, our visualization allows the end-user to navigate the network with an interactive navigation panel, but also integrates published materials within the visualization interface.

%\section{General and specific Objectives}
%The purpose of this research is to analyze the structure and dynamics of scientific collaborations and co-authorship in the fields of Malaria, Tuberculosis and HIV/AIDS research areas over the last 20 years in the Republic of Benin. Our results can help improve grant and research resource allocation to funding and help research organizations and national control programs to promote and encourage trans and interdisciplinary research in the country. Additionally, our findings recommend new approaches to support the Beninese national control programs via better strategic planning and implementation of public health policies, research and development. We also propose a prototype of an online research collaboration tool to assist health policy makers and funding organizations to promote research collaboration in the republic of Benin. More specifically, we address the following research questions:
%\begin{itemize}
%	\item What is the structure of scientific research collaboration networks in Benin over the last 20 years in Malaria, TB and HIV/AIDS research?
%	\item Who are the most prolific authors, scientific research groups within each field?
%	\item How have transnational research evolved over the last two decades in the Republic of Benin?
%	\item What are the characteristics and the dynamics of the current co-authorship research collaborations in Benin in Malaria, TB and HIV/AIDS research?
%\end{itemize}
%This dissertation fills the gap in the current literature, and reveals the role of the collaborative research in the prevailing research networks. Our research meets the following specific objectives:
%\begin{enumerate}
%	\item To identify the most productive and prolific scientific research groups and authors within each research area.
%	\item To document and describe the structure of Malaria, TB, HIV/AIDS co-authorship networks and their characteristics, how they evolve over time in Benin over the last two decades.
%	\item To unravel the mechanistic phenomenon explaining the formation and trends of these networks over time.
%	\item To predict and recommend future research collaboration ties in Benin in the three research areas.
%	\item To develop a prototype of co-authorship visualization and scientific collaboration tool for Malaria, TB and HIV/AIDS research in Benin.
%\end{enumerate}	
%
%\section{Hypotheses}
%\label{hypotheses}
%We hypothesize that tie formation in each co-authorship network:
%\begin{itemize}
%\item is dependent on observed authors (vertices) characteristics
%\item is dependent on the concept of distance in latent space, and
%\item is dependent on collaboration types and/or membership to a certain research community or cluster.
%\end{itemize}
%\section{Gap in the literature}
%Despite the increasing financing effort and increasing number of published reports, the literature does not provide sufficient data regarding co-authorship networks of scientific research collaborations and their dynamics in the fields of malaria and TB and HIV/AIDS research in Africa, and particularly in Benin. Knowing such information is crucial to consolidate the progress made at controlling those diseases, support cooperative and translational research initiatives \cite{gonzalez-alcaide_scientific_2012}. Lack of such information makes it difficult for policy makers to sustain the important progress made to reduce morbidity and mortality rates.
