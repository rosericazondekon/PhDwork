%*******10********20********30********40********50********60********70********80

% For all chapters, use the newdefined chap{} instead of chapter{}
% This will make the text at the top-left of the page be the same as the chapter
\clearpage  % Start a new page
\lhead{\emph{General Introduction}}  % left side page header to "List of Tables"
\chapter*{\centering General Introduction}
%\chap{General Introduction}
\addcontentsline{toc}{chapter}{General Introduction}
Infectious diseases have long claimed the lives of millions of people worldwide. They disproportionately affect the developing nations where 90\% of the deaths are caused by very few diseases among which Malaria, Tuberculosis (TB) and HIV/AIDS \cite{davis_emerging_2001}. Malaria, TB and HIV/AIDS remain the three major public health concerns in Sub Saharan Africa where they are responsible for high mortality, morbidity rates and impact negatively on the socioeconomic way of life of the populations \cite{gallup_economic_2001,vitoria_global_2009}. These three diseases have been given special attention at the Millenium Declaration in its $6^{th}$ Goal of Millenium Development \cite{assembly_united_2000}. Initiatives such as the US President’s Malaria Initiative, the Global Fund for Malaria, TB and HIV/AIDS and the President’s Emergency Plan for AIDS have led to the investment of more than 70 million of US dollars to encourage Research and Development, Private-Public partnership as well as to reinforce the activities of non-governmental organizations within the healthcare systems of the affected countries \cite{arthur_institute_2014,murray_global_2014,stoops_presidents_2008}.\\
The Global Fund disbursement in 2010 peaked at over 1.45 billion dollars for HIV/AIDS, 416 million dollars for TB and 714 million dollars for Malaria \cite{global_fund_making_2011,world_health_organization_world_2012}. With these financial supports at hand, efforts have led to a sharp increase of public health interventions and many positive public health outcomes in terms of the reduction of mortality and morbidity related to those diseases \cite{barat_four_2006}. For example, in Benin, such increase in public health interventions translated in the financing, successful implementation and sustainability of the entomological surveillance of malaria for more than six years since 2008 \cite{akogbeto_six_2015}.  Encouraged and motivated by the success stories in controlling these diseases, some authors formulated the ambitious zero incidence goal of TB and HIV and the zero death goal of the three diseases by 2015 \cite{joint_united_nations_programme_on_hiv/aids_getting_2010}. \\
After the declaration of the Millenium Challenge Goal 6 in 2000, significant progress has been made in the treatment and prevention of Malaria, TB and HIV/AIDS, leading to the reverse of the mortality and morbidity due to these three diseases. Nevertheless, sub Saharan Africa still carries the burden of these diseases. For example, in 2009, 2.6 million new cases and 1.8 million of death related to HIV were estimated out of which 68\% and 72\% of respectively new cases and deaths were in Africa \cite{joint_united_nations_programme_on_hiv/aids_global_2010}. TB cases were estimated at 9.4 million and 1.3 million deaths out of which HIV-positive cases make up 12\% of all cases and 23\% of all TB deaths \cite{world_health_organization_global_2010}. Although the rapid expansion of vector control strategies worldwide, malaria was responsible of 225 million cases and 781,000 death in 2009 out of which over 90\% were in Africa \cite{world_health_organization_world_2012}. \\
In the Republic of Benin, between 2000 and 2013, the impact of the increase in funding has led to an annual decrease in the incidence of 7.6\%, 0.6\% and 5.2\% respectively in HIV/AIDS, TB and Malaria. Similar results were obtained in terms of prevalence with a decrease of 1.3\% in HIV/AIDS and 0.8\% in TB. Annual death rates decreased also at about 3.1\%, 1.2\% and 5.3\% respectively in HIV/AIDS, TB and Malaria \cite{world_health_organization_world_2012,joint_united_nations_programme_on_hiv/aids_global_2010,world_health_organization_global_2010}.\\
Successful scientific collaborations have led to the eradication of chickenpox and the near eradication of poliomyelitis through the development of vaccines \cite{jamison_disease_2006}. For Malaria and HIV/AIDS, the development of a vaccine has proven significantly difficult to develop despite the decades of active research that has not been successful so far \cite{long_malaria_2016,titti_problems_2007,walker_toward_2008}. This is why researchers need to form continuous and sustainable collaborations through intensive network practices that go beyond the regional boundaries \cite{newman_structure_2001}. Scientific collaborations give researchers the opportunity to work and learn from each other. Such collaborations are further needed to overcome the overgrowing challenge of co-infections of HIV/AIDS and Tuberculosis \cite{corbett_growing_2003,gandhi_hiv_2010}. In the republic of Benin, Malaria, TB and HIV/AIDS have become a common aspect of the public health system. The three are the main impediments of economic and social progress that are characteristics of poverty. According to a 2000 World Health Organization (WHO) press report, malaria slows economic growth on the African continent by 1.3\% each year \cite{world_health_organization_economic_2000}. And it is known that Tuberculosis and HIV/AIDS patients experienced severe economic burden in terms of access to health care, treatment and diagnosis \cite{richter_economic_2014}. The situation is further compounded by the poorly developed immunity among the children and the elderlies, and the predominant malnutrition problem experienced by a majority of the population \cite{jamison_disease_2006}. The disappointing aspect is that the extensive research conducted has not prevented these three diseases from outpacing the proposed solutions and the progress made \cite{akukwe_dont_2006}.\\
Therefore, in this thesis to document, we document, describe and analyze the different aspects of scientific research collaboration of the three leading infectious diseases in the Republic of Benin. The social network analysis of research collaboration approach is chosen to reveal undiscovered knowledge on effort of researchers in working together towards the reduction of the burden of Malaria, TB and HIV/AIDS. Modern times have rendered research and scientific collaborations irreplaceable policy formulations processes. This is because research collaboration forms a stable basis for the provision of evidence based information in the formulation of fundamental principles and guidelines for the elaboration of public health strategies, particularly in developing countries like Benin. For this reason, this thesis focuses on the Network analysis of the scientific collaborations through co-authorship network analysis.
