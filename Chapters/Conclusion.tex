%*******10********20********30********40********50********60********70********80

% For all chapters, use the newdefined chap{} instead of chapter{}
% This will make the text at the top-left of the page be the same as the chapter
\clearpage  % Start a new page
%\lhead{\emph{General Conclusion}}  % left side page header to "List of Tables"
%\chapter*{\centering General Conclusion}
\chap{General Conclusion}
%\addcontentsline{toc}{chapter}{General Conclusion}
In this dissertation, we have documented and described the collaborative pattern in Malaria, HIV/AIDS and TB research in Benin. Our findings suggest that each one of the collaborative research network of Malaria, HIV/AIDS and TB has a complex structure. We modeled these complex structures to predict the establishment of future collaboration ties. We implemented the models in a shiny-based application for co-authorship visualization and scientific collaboration prediction tool which we named \textbf{AuthorVis}.

\subsection*{Strengths and Limitations}
The application of temporal or dynamic modeling techniques is the major strength of our research along with its application of not only descriptive methods but also robust network analysis methods such as inferential methods like Monte-Carlo simulations, unlike most studies on co-authorship analysis. Our data mining strategy involved a robust machine learning algorithm that helped address the crucial issue of the disambiguation of authors names and assigns a unique identifier to each of them. To the best of our knowledge, our study is the first to describe the malaria research collaborations network via co-authorship network analysis in Benin. It is also the first to apply statistical network models to investigate co-authorship networks in a specific research area in an African country. \\%~\\
The fact that we collected data only from the Web Of Science can be considered as an important limitation of this study. However, according to Falagas and colleagues \cite{falagas_comparison_2007}, who compared PubMed, Scopus, Web Of Science and Google Scholar in their paper, the Web Of Science appears as a reasonable scientific database source for our analysis. In addition, it proved to cover a wide range of both old and recently published papers. Falagas and colleagues \cite{falagas_comparison_2007} found PubMed to be the optimal choice in terms of scientific database. For that reason we ran the same bibliographic searches in PubMed. Unfortunately, the Web Of Science returns more relevant data than PubMed. \\
Another major limitation is related to the manual curation of the scientific publications and the keyword based searches of the literature involved in this study. It is therefore worth acknowledging the possibility of error or incompleteness of the scientific publications reviewed. However, we limited this possibility by casting a wider net, querying the Web Of Science API with wider keywords, then narrowing the search down by combining the keywords. Yet another major limitation is that only one manual curator has reviewed the publications for the selection criteria. Having multiple curators would have allowed us to evaluate the quality of the search by measuring selection agreement   statistics (kappa statistic for example) between the curators.\\
The nature of all co-authorship studies in itself is another limitation of this study. Collaborators, in co-authorship networks, do not often come from the same scientific discipline, or do not play the same roles on a particular research project. The data we collected did not allow us to accurately assess or even infer the disciplines each author comes from or their specific contribution in the published documents.

\subsection*{Future Directions}
There are several future directions. Our work can be extended to the entire African collaboration network in Malaria, HIV/AIDS and TB. Since collaborations usually are often initiated between individuals, labs or even countries, the analysis of bipartite co-authorship networks is an interesting direction to our study. \\
Currently, \textbf{AuthorVis} is specifically built for Malaria, TB and HIV/AIDS in Benin. Future developments may extend the tool to other research domain. Adding a general purpose module to \textbf{AuthorVis} for the visualization of any user-input co-authorship network is an interesting venture since it will also require the integration of a data pre-processing module to facilitate the disambiguation and deduplication of co-authorship information. Furthermore, incorporating a layered structured network visualization \cite{nakazono_nel_2006} functionality to the visualization in order to display temporal changes in the evolution of the co-authorship network is another interesting direction. It can, in addition be designed into a real-time, cross-domain, and cross-collection co-authorship visualization interface capable of automatically searching the literature. \\
Outside of the realm of co-authorship analyses, the same idea of network analyses and visualization can be extended to other important disciplines such as Neuroscience. In analogy to co-authorship networks, the brain functioning can be represented as a brain connectivity network (connectome) where parcels or anatomical regions or regions of interest of the brain represent the vertices and the edges determine statistical dependency of combined neuronal activities between the vertices. \\
Basic network analyses have already enabled the development of network-based clinical diagnostics of certain pathologies such as schizophrenia \citep{LynallFunctionalConnectivityBrain2010}, stroke \citep{GrefkesReorganizationcerebralnetworks2011}, and Alzheimer’s disease \citep{TijmsAlzheimerdiseaseconnecting2013}. Although trending, modeling brain connectivity networks by means of the methods used in this dissertation remains limited to very few studies  \citep{SimpsonExponentialRandomGraph2011,Forerostatisticalmodelbrain2017,ForeroGraphModelsBrain2015,DeVicoFallaniGraphanalysisfunctional2014,WangExponentialrandomgraph2013,SinkeBayesianexponentialrandom2016} in neuroscience. Since it is important to better explain the functional organization of the brain and to allow inference of specific brain properties, the visualization of real time brain connectivity dynamics has potentials for the development of Brain Computer Interfaces. See appended \hyperlink{app:draft}{neuroscience manuscript draft}. %Future research ventures taking into account these suggestions will be useful
% Extend this work to all the entire African research network in the three domains of research and build a link prediction and recommendation model
% Build a general purpose co-authorship network visualization with user input co-authorship networks.
% Extend this work to designing a real time, cross-domain, and cross-collection co-authorship visualization interface capable of automatically searching the literature
% The same idea of graph visualization can be extended to visualizing real time connectivity dynamics between parcels of the brain during certain resting, memory or motor tasks.
% NOT SURE!!!:===> The same methods used in this dissertation have already been suggested in neuroscience where one of the first model of network were reported from. While certain studies have already attempted modeling brain connectivity using advance statistical network models such as ERGM, the scope of these studies remain limited in terms of the neuroimaging data used.

%Our first approach to modeling our network relied on the use of SBM. In addition of being a model based clustering method, the SBM identified important organizational and interactional patterns in the network. It is worth noting that many of the studies on co-authorship network analysis are descriptive in nature. This study is one of the rare co-authorship network analysis to model a co-authorship network using advanced statistical models. ERGM is the leading approach to modeling network \cite{schmid_exponential_2017}. The literature has reported application of this model in studying various social network such as the analysis of friendship and obesity \cite{valente_adolescent_2009,de_la_haye_obesity-related_2010}, the exploration of the association between hormone and social network structure \cite{kornienko_hormones_2014}. Similarly to friendship networks, the use of ERGM to model co-authorship networks is easily justified. However, the size of our network prevented the fitting of complex models including dyadic and structural terms. In addition, our best model failed to adequately fit the observed network data. This lack of goodness-of fit, according to Hunter, Goudreau and Handcock \cite{hunter_goodness_2008}, could be improved by including the geometrically weighted edgewise shared partner, geometrically weighted dyadic shared partner, and geometrically weighted degree network statistics to our model. Although, we follow such recommendations by including these structural network statistics to our final model, the ERGM model failed to converge after a maximum of 1,000 iterations. Furthermore, at about 750 iterations, we notice that the processing became both computationally intensive and expensive in terms of CPU time and memory usage.  In a recently published paper, Schmid and Desmarais \cite{schmid_exponential_2017} acknowledged the difficulty of fitting network which size is of the order 1,000 vertices using ERGM. They recommended that using the maximum pseudolikelihood estimation (MPLE) instead of the Monte Carlo maximum likelihood (MCMLE) could tremendously reduce computation time. Having followed these recommendations too, the model containing dyadic and structural terms still failed to converge. Let's recall that our weighted co-authorship network contains 1,792 vertices for 95,707 edges. We suspect that the number of edges, the large size of the network added to the possibility of hidden/latent variables might justify the failure of the final model containing the dyadic and structural terms to converge. We remedy this situation by applying LNM to the observed network data. \\
