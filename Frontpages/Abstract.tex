% The Abstract Page
\clearpage
\pagestyle{plain}
\cfoot{\thepage}
%\lhead{}  % Set Left page header to nothing.

%\addtotoc{Abstract}  % Add the "Abstract" page entry to the Contents
\begin{center}
	\textbf{ABSTRACT}\\~\\
	\MakeUppercase{Network Analysis of Scientific Collaboration and Co-authorship of the Trifecta of Malaria, Tuberculosis and HIV/AIDS in Benin.}\\~\\
by\\~\\
Gbedegnon Roseric Azondekon\\~\\
The University of Wisconsin-Milwaukee, 2018\\
Under the Supervision of Professor Susan McRoy
\end{center}
\setstretch{2}
%\abstract{
    %\addtocontents{toc}{\vspace{0em}}  % Add a gap in the Contents, 
                                        %for aesthetics
    %The Thesis Abstract is written here (and usually kept to just this page). 
    %The page is kept centered vertically so can expand into the blank space above the title too \ldots
    Despite the international mobilization and increase in research funding, Malaria, Tuberculosis and HIV/AIDS are three infectious diseases that have claimed more lives in sub Saharan Africa than any other place in the World. Consortia, research network and research centers both in Africa and around the world team up in a multidisciplinary and transdisciplinary approach to boost efforts to curb these diseases. Despite the progress in research, very little is known about the dynamics of research collaboration in the fight of these Infectious Diseases in Africa resulting in a lack of information on the relationship between African research collaborators. This dissertation addresses the problem by documenting, describing and analyzing the scientific collaboration and co-authorship network of Malaria, Tuberculosis and HIV/AIDS in the Republic of Benin.\\
    We collected published scientific records from the Web Of Science over the last 20 years (From January 1996 to December 2016). We parsed the records and constructed the coauthorship networks for each disease. Authors in the networks were represented by vertices and an edge was created between any two authors whenever they coauthor a document together. We conducted a descriptive social network analysis of the networks, then used mathematical models to characterize them. We further modeled the complexity of the structure of each network, the interactions between researchers, and built predictive models for the establishment of future collaboration ties. Furthermore, we implemented the models in a shiny-based application for co-authorship network visualization and scientific collaboration link prediction tool which we named \textbf{AuthorVis}.\\
    %We further applied complex statistical network modeling methods to model the underlying phenomenon influencing collaboration tie establishments.\\
    Our findings suggest that each one of the collaborative research networks of Malaria, HIV/AIDS and TB has a complex structure and the mechanism underlying their formation is not random. All collaboration networks proved vulnerable to structural weaknesses. In the Malaria coauthorship network, we found an overwhelming dominance of regional and international contributors who tend to collaborate among themselves. We also observed a tendency of transnational collaboration to occur via long tenure authors. We also find that TB research in Benin is a low research productivity area. We modeled the structure of each network with an overall performance accuracy of $79.9\%$, $89.9\%$, and $93.7\%$ for respectively the malaria, HIV/AIDS, and TB coauthorship network. \\ 
    Our research is relevant for the funding agencies operating and the national control programs of those three diseases in Benin (the National Malaria Control Program, the National AIDS Control Program and the National Tuberculosis Control Program). %Our tool prototype can help improve grant and research prioritization and resource allocation to funding and help policy makers ensure in the country.
    %Because multidisciplinary and transdisciplinary research approaches have been proven successful in achieving sound and robust findings, we believe that our research is crucial for the future of research funding in Benin and in Africa.
%Other studies have already reported a universal rise in terms of scientific collaborations.  Understanding the structure of these complex networks is capital since it can help improve research prioritization, identification of prolific researchers, better design, strategic planning and implementation of research program, and promote cooperation and translational research initiatives. In this doctoral thesis proposal, we propose to document, describe and analyze the scientific collaboration, and co-authorship of the research conducted in the Republic of Benin on Malaria, Tuberculosis and HIV/AIDS.% Our strategy consists in mining the literature and tracking the scientific papers published in the available scientific database over the last 20 years (From January 1996 to December 2016). Our research is relevant for the funding agencies operating in Benin and the different national control programs of those three diseases in Benin (the National Malaria Control Program, the National AIDS Control Program and the National Tuberculosis Control Program). Our findings will help improve grant and research prioritization and resource allocation to funding and help research organizations as well as national control programs to promote and encourage transdisciplinary and interdisciplinary research in the country. In addition, our results will recommend new approaches and important tools to support the Beninese national control programs via better strategic planning and implementation of public health policies, research and development.  Because multidisciplinary and transdisciplinary research approaches have been proven successful in achieving sound and robust findings, we believe that our research is crucial for the future of research funding in Benin and in Africa. This is why, the last focus of our proposal is the prototyping and evaluation of an online, real-time research collaboration tool to help researchers, governmental agencies and funding organizations promote cooperation and translational research initiative in the republic of Benin.
%}

