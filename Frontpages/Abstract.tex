% The Abstract Page
\clearpage 
\setstretch{1.1}
\addtotoc{Abstract}  % Add the "Abstract" page entry to the Contents
\abstract{
    \addtocontents{toc}{\vspace{1em}}  % Add a gap in the Contents, 
                                        %for aesthetics
    %The Thesis Abstract is written here (and usually kept to just this page). 
    %The page is kept centered vertically so can expand into the blank space above the title too \ldots
    Despite the international mobilization and increase in research funding, Malaria, Tuberculosis and HIV/AIDS are three infectious diseases that have claimed more lives in sub Saharan Africa than any other place in the World. Meanwhile, research collaborations have peaked in an ultimate effort to dramatically decrease the mortality and morbidity of those diseases on the continent. Consortia, research network and research centers both in Africa and around the world team up in a multidisciplinary and transdisciplinary approach to boost efforts to curb these diseases. Other studies have already reported a universal rise in terms of scientific collaborations. Despite the progress in research, very little is known on the dynamics of research collaboration in the fight of these Infectious Diseases in Africa resulting in a lack of information on the relationship between African research collaborators. Understanding the structure of these complex networks is capital since it can help improve research prioritization, identification of prolific researchers, better design, strategic planning and implementation of research program], and promote cooperation and translational research initiatives. In this doctoral thesis proposal, we propose to document, describe and analyze the scientific collaboration, and co-authorship of the research conducted in the Republic of Benin on Malaria, Tuberculosis and HIV/AIDS.% Our strategy consists in mining the literature and tracking the scientific papers published in the available scientific database over the last 20 years (From January 1996 to December 2016). Our research is relevant for the funding agencies operating in Benin and the different national control programs of those three diseases in Benin (the National Malaria Control Program, the National AIDS Control Program and the National Tuberculosis Control Program). Our findings will help improve grant and research prioritization and resource allocation to funding and help research organizations as well as national control programs to promote and encourage transdisciplinary and interdisciplinary research in the country. In addition, our results will recommend new approaches and important tools to support the Beninese national control programs via better strategic planning and implementation of public health policies, research and development.  Because multidisciplinary and transdisciplinary research approaches have been proven successful in achieving sound and robust findings, we believe that our research is crucial for the future of research funding in Benin and in Africa. This is why, the last focus of our proposal is the prototyping and evaluation of an online, real-time research collaboration tool to help researchers, governmental agencies and funding organizations promote cooperation and translational research initiative in the republic of Benin.
}

